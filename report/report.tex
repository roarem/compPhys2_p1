\documentclass[12pt]{article}

\usepackage{listings}
\usepackage{amsmath}
\usepackage{hyperref}
\usepackage{bm}

\lstdefinestyle{c++}{
basicstyle=\scriptsize\ttfamily,
frame=single,
}
\begin{document}

\section{Method}

  The Hamiltonian for the we are using has the following form


  where the external potential given by the boson trap


  and a repulsive potential due to bosons interaction given by

  As for the trial wavefunction for the ground state with $N$ atoms


  with $\alpha, \beta$ as variational parameters. The correlation function 
  $f(a,|{\bf r}_i-{\bf r}_j|)$ is given by

 
  \paragraph{Analytical local energy $E_L$}
   
  Using natural units and setting the interaction potential to zero and $\beta = 1$,
  $\Psi_T$ becomes $exp[-\alpha r^2]$. The second derivative will then be
  \begin{equation}
    \nabla^2\Psi_T = \nabla^2 e^{-\alpha r^2} = \nabla -\alpha 2r e^{-\alpha r^2}%
    = \alpha e^{-\alpha r^2}(1 - 2\alpha r^2)
  \end{equation}
  %
  Giving us the expression for the local energy
  %
  \begin{equation}
    E_L = \frac{\nabla^2 \Psi_T}{\Psi_T} = \alpha (1 - 2\alpha r^2)
  \end{equation}

  \section{Introduction}
  Since it has been shown that Bose-Einstein condensation occur in gases of alkali 
  atoms confined in magnetic traps, understanding the confined Bose systems has 
  become a popular field. In this project I have tried to calculate the ground 
  state energy of various number of trapped bosons using Variational Monte Carlo 
  calculations with a specific trial wavefunction. Firstly using the spherical 
  harmonic oscillator as a trap with no interaction between the particles, then
  swtiching to an elliptical trap. Where in both cases I calculate both the 
  analytical and numerical solution. Also running the calculations with and 
  without a repulsive potential due to the fact that bosons have a size.

  \section{Method}
  \subsection{System}
  A two-body Hamiltonian of this system is on the form
  \begin{equation}
     H = \sum_i^N \left(\frac{-\hbar^2}{2m}\nabla^2_i + V_{ext}(\bm{r}_i)\right) + %
     \sum_{i< j}^N V_{int}(\bm{r}_i,\bm{r}_j)
  \end{equation}
  where the external potential for the trap is for the spherical and elliptical part
  given by
  \begin{equation}
    V_{ext}(\bm{r}) = \left\{%
      \begin{array}{lr}
	\frac{1}{2}m\omega^2r^2 & \text{Spherical}\\
	\frac{1}{2}m[\omega^2(x^2 + y^2) + \gamma^2z^2] & \text{Elliptical}
      \end{array}
      \right.
  \end{equation}
  where $\omega$ is the trap potential strength, and in the elliptical trap $\gamma$ is
  the strength in z-direction.
  The internal potential which represents the repulsion when two boson gets close is
  \begin{equation}
    V_{int}(|\bm{r}_i - \bm{r}_j|) = \left\{%
    \begin{array}{lr}
      \infty & |\bm{r}_i - \bm{r}_j| \leq a\\
      0 & |\bm{r}_i - \bm{r}_j| > a
    \end{array}
    \right.
  \end{equation}
  with $a$ as the hard-core diamater of the bosons.
  Setting up the system with a trial wavefunction $\Psi_T$ for the ground state of the form
  \begin{equation}
    \Psi_T({\bf R})=\Psi_T({\bf r}_1, {\bf r}_2, \dots {\bf r}_N,\alpha,\beta)=\prod_i%
    g(\alpha,\beta,{\bf r}_i)\prod_{i<j}f(a,|{\bf r}_i-{\bf r}_j|),
  \end{equation}
  using $\alpha$ and $\beta$ as variational parameters, and 
  \[ g(\alpha, \beta, \bm{r}_i) = e^{-\alpha(x_i^2 + y_i^2 + \beta z_i^2)} \]
  and
  \begin{equation}
    f(a,|{\bf r}_i-{\bf r}_j|)=\Bigg\{
    \begin{array}{ll}
      0 & {|{\bf r}_i-{\bf r}_j|} \leq {a}\\
	(1-\frac{a}{|{\bf r}_i-{\bf r}_j|}) & {|{\bf r}_i-{\bf r}_j|} > {a}.
    \end{array}
  \end{equation}  
  %
  For the simplest cases where I set the boson size $a = 0$ and $\beta = 1$ the 
  trial wavefunction becomes 
  \begin{equation}
    \Psi_T(\bm{R}) = e^{-\alpha r^2}
    \label{eq:psi1}
  \end{equation}
  Calculating the double derivative of the wavefunction returns
  \begin{equation}
    \nabla^2\Psi_T = \nabla^2 e^{-\alpha r^2} = \nabla -\alpha 2r e^{-\alpha r^2}%
    = \alpha e^{-\alpha r^2}(1 - 2\alpha r^2)
  \end{equation}
  then inserting this result into the expression for the local energy yields
  \begin{equation}
    E_L(\bm{R}) = \frac{1}{\Psi_T(\bm{R})}H\Psi_T(\bm{R}) = \alpha(1-2\alpha r^2)
  \end{equation}
  %
  which is the analytical experssion for the local energy in a spherical trap 
  without interaction.
  For a simple test case I use 1 dimension, 1 particle, $\alpha = 1/2$ and position $x=0$ 
  %
  \[E_L (0) = \frac{1}{2}(1-2\frac{1}{2}0) = \frac{1}{2}\]
  %
  \subsection{Numerical method}
  For the first part I use a Variational Monte Carlo program with a brute force Metropolis
  sampling to calculate the ground state energy. This is done by first placing the system in
  a random, Gaussian distributed position around 0 with a $\sigma = 1/\sqrt{2}$. Evaluating the
  wavefunction according equation \ref{eq:psi1}. Then choosing a random particle and dimension,
  and moving it with a Gaussian distribution to a new position. Do the same evaluation for
  the new position and comparing
  \[\frac{|\Psi_{New}|^2}{|\Psi_{Old}|^2}\]
  Then by taking a random, uniformly distributed number between 0 and 1 and compare it
  to the ratio between the new and old wavefunction. If the ratio is larger I accept
  the new step, if not I revert back to the old position. Then lastly I sample the energy for
  the resulting system.
  \lstinputlisting[caption=Brute force Metropolis, style=c++]{codes/bruteForceMetro.cpp}
  When using this method I get the result which is shown in table \ref{h:a1} and is
  in correspondence with what is expected from the analytical solution.
  
  \newpage
  \appendix
  \newpage
\section{Tables}
\subsection{Brute force Metropolis algorithm}
All runs are with with $10^5$ cycles and step length
of 1.7.
%\subsubsection{Analytical 1D}
%\begin{table}[h!]
\centering 
\begin{tabular}{|l|l|l|l|l|}
\hline 
N particles & $<E>$ & Variance & Accepted & Time [s]\\ 
 \hline 
1 & 5.000000e-01 & 0.000000e+00 & 0.553895 & 0.018 \\ \hline 
10 & 5.000000e+00 & 0.000000e+00 & 0.548784 & 0.024 \\ \hline 
100 & 5.000000e+01 & 0.000000e+00 & 0.550417 & 0.075 \\ \hline 
500 & 2.500000e+02 & 0.000000e+00 & 0.550962 & 0.312 \\ \hline 
\end{tabular}
\label{tab:ha1} 
\end{table} 

\subsubsection{Analytical 2D}
\begin{table}[h!]
\begin{tabular}{|l|l|l|l|l|}
\hline 
N particles & $<E>$ & Variance & Accepted & Time [ms]\\ 
 \hline 
1 & 1.000000e+00 & 0.000000e+00 & 0.968367 & 470 \\ \hline 
10 & 1.000000e+01 & 0.000000e+00 & 0.967963 & 711 \\ \hline 
100 & 1.000000e+02 & 0.000000e+00 & 0.968089 & 2808 \\ \hline 
500 & 5.000000e+02 & 0.000000e+00 & 0.968799 & 12433 \\ \hline 
\end{tabular}
\label{h:a2} 
\end{table}

\subsubsection{Analytical 3D}
\begin{tabular}{|l|l|l|l|l|}
\hline 
\multicolumn{5}{|c|}{Analytical}\\ 
\hline 
N particles & $<E>$ & Variance & Accepted & Time [ms]\\ 
 \hline 
1 & 1.500000e+00 & 0.000000e+00 & 0.968941 & 484 \\ 
\hline10 & 1.500000e+01 & 0.000000e+00 & 0.967473 & 792 \\ 
\hline100 & 1.500000e+02 & 0.000000e+00 & 0.968197 & 3623 \\ 
\hline500 & 7.500000e+02 & 0.000000e+00 & 0.969128 & 16629 \\ 
\hline\end{tabular}
%\subsubsection{Numerical 1D}
%\begin{table}[h!]
\centering 
\begin{tabular}{|l|l|l|l|l|}
\hline 
N particles & $<E>$ & Variance & Accepted & Time [s]\\ 
 \hline 
1 & 5.000000e-01 & -9.436896e-16 & 0.550639 & 0.023 \\ \hline 
10 & 5.000000e+00 & -5.115908e-13 & 0.548139 & 0.099 \\ \hline 
100 & 5.000000e+01 & 3.092282e-11 & 0.551551 & 2.868 \\ \hline 
500 & 2.500000e+02 & 3.419700e-10 & 0.550273 & 61.191 \\ \hline 
\end{tabular}
\caption{%
	  Benchmark of the brute force Metropolis method with %
	  numerical calculation of the Laplacian. Using $10^5$ %
	  cycles and a step length of 1.7.%
	}
\label{tab:hn1} 
\end{table} 

\newpage
\subsubsection{Numerical 2D}
\begin{tabular}{|l|l|l|l|l|}
\hline 
\multicolumn{5}{|c|}{Numerical 2D}\\ 
\hline 
N particles & $<E>$ & Variance & Accepted & Time [ms]\\ 
 \hline 
1 & 9.999999e-01 & 6.494805e-14 & 0.968163 & 751 \\ \hline 
10 & 9.999999e+00 & 1.747935e-12 & 0.968789 & 5478 \\ \hline 
100 & 9.999999e+01 & 1.618901e-10 & 0.968061 & 235928 \\ \hline 
500 & 5.000000e+02 & 2.732850e-08 & 0.969589 & 5144598 \\ \hline 
\label{h:n2} 
\end{tabular}
\subsubsection{Numerical 3D}
\begin{tabular}{|l|l|l|l|l|}
\hline 
\multicolumn{5}{|c|}{Numerical}\\ 
\hline 
N particles & $<E>$ & Variance & Accepted & Time [ms]\\ 
 \hline 
1 & 1.500000e+00 & 1.509903e-14 & 0.968088 & 917 \\ 
\hline100 & 1.500000e+02 & 2.619345e-10 & 0.967844 & 415179 \\ 
\hline100 & 1.500000e+02 & -7.275958e-11 & 0.968416 & 413230 \\ 
\hline500 & 7.499999e+02 & -3.352761e-08 & 0.969971 & 9641421 \\ 
\hline\end{tabular}
\newpage
\subsection{Metropolis algorithm with Importance sampling}
All runs are with with $10^5$ cycles and step length
of 0.05.
%\subsubsection{Analytical 1D}
%\begin{table}[h!]
\begin{tabular}{|l|l|l|l|l|}
\hline 
N particles & $<E>$ & Variance & Accepted & Time [s]\\ 
 \hline 
1 & 5.000000e-01 & 0.000000e+00 & 0.998522 & 0.03 \\ \hline 
10 & 5.000000e+00 & 0.000000e+00 & 0.998700 & 0.038 \\ \hline 
100 & 5.000000e+01 & 0.000000e+00 & 0.998644 & 0.111 \\ \hline 
500 & 2.500000e+02 & 0.000000e+00 & 0.998922 & 0.451 \\ \hline 
\end{tabular}
\label{tab:ia1} 
\end{table} 

\subsubsection{Analytical 2D}
\begin{table}[h!]
\begin{tabular}{|l|l|l|l|l|}
\hline 
N particles & $<E>$ & Variance & Accepted & Time [s]\\ 
 \hline 
1 & 1.000000e+00 & 0.000000e+00 & 0.998744 & 0.03 \\ \hline 
10 & 1.000000e+01 & 0.000000e+00 & 0.998644 & 0.044 \\ \hline 
100 & 1.000000e+02 & 0.000000e+00 & 0.998689 & 0.181 \\ \hline 
500 & 5.000000e+02 & 0.000000e+00 & 0.998800 & 0.817 \\ \hline 
\end{tabular}
\label{tab:ia2} 
\end{table} 

\subsubsection{Analytical 3D}
\begin{tabular}{|l|l|l|l|l|}
\hline 
\multicolumn{5}{|c|}{Analytical}\\ 
\hline 
N particles & $<E>$ & Variance & Accepted & Time [ms]\\ 
 \hline 
1 & 1.500000e+00 & 0.000000e+00 & 0.996373 & 780 \\ 
\hline10 & 1.500000e+01 & 0.000000e+00 & 0.996321 & 1198 \\ 
\hline100 & 1.500000e+02 & 0.000000e+00 & 0.996450 & 5396 \\ 
\hline500 & 7.500000e+02 & 0.000000e+00 & 0.996357 & 24110 \\ 
\hline\end{tabular}
%\subsubsection{Numerical 1D}
%\begin{table}[h!]
\begin{tabular}{|l|l|l|l|l|}
\hline 
N particles & $<E>$ & Variance & Accepted & Time [ms]\\ 
 \hline 
1 & 4.999999e-01 & 1.915135e-15 & 0.996353 & 839 \\ \hline 
10 & 4.999999e+00 & -2.351896e-12 & 0.996342 & 2724 \\ \hline 
100 & 4.999999e+01 & 3.310561e-10 & 0.996470 & 68701 \\ \hline 
500 & 2.500000e+02 & 5.456968e-10 & 0.996473 & 1410677 \\ \hline 
\end{tabular}
\label{i:n1} 
\end{table}

\subsubsection{Numerical 2D}
\begin{table}[h!]
\begin{tabular}{|l|l|l|l|l|}
\hline 
N particles & $<E>$ & Variance & Accepted & Time [s]\\ 
 \hline 
1 & 9.999999e-01 & 1.099121e-14 & 0.998756 & 0.041 \\ \hline 
10 & 9.999999e+00 & 6.963319e-13 & 0.998767 & 0.223 \\ \hline 
100 & 9.999999e+01 & 2.546585e-11 & 0.998711 & 10.117 \\ \hline 
500 & 4.999999e+02 & -5.587935e-09 & 0.998789 & 232.673 \\ \hline 
\end{tabular}
\label{tab:in2} 
\end{table} 

\subsubsection{Numerical 3D}
\begin{tabular}{|l|l|l|l|l|}
\hline 
\multicolumn{5}{|c|}{Numerical}\\ 
\hline 
N particles & $<E>$ & Variance & Accepted & Time [ms]\\ 
 \hline 
1 & 1.500000e+00 & 1.243450e-13 & 0.996377 & 1198 \\ 
\hline10 & 1.500000e+01 & 6.025402e-12 & 0.996434 & 9108 \\ 
\hline100 & 1.500000e+02 & 1.615263e-09 & 0.996397 & 429843 \\ 
\hline500 & 7.499999e+02 & -9.709038e-08 & 0.996334 & 9671595 \\ 
\hline\end{tabular}

\end{document}
